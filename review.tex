\documentclass[fleqn,10pt]{wlscirep}
\usepackage[utf8]{inputenc}
\usepackage{amsfonts,amssymb,amsmath}
\usepackage{graphicx}


\title{An Interdisciplinary Approach to Morphogenesis}
% alternative epic title : Morphogenesis : a Tale of Life and Death

\author[1]{Chenling Antelope}
\author[2]{Lars Hubatsch}
\author[3]{Juste Raimbault}
\author[4]{Jesus Mario Serna}
% E. Crosato : Emanuele may contribute later but does not want to be in 

\affil[1]{Affiliation, department, city, postcode, country}
\affil[2]{Affiliation, department, city, postcode, country}

\affil[*]{corresponding.author@email.example}

%\affil[+]{these authors contributed equally to this work}


%\date{July 6th, 2016}

%\keywords{Keyword1, Keyword2, Keyword3}

%\maketitle


\begin{abstract}
The notion of morphogenesis seems to play an important role in the study of a broad range of complex systems. If the concept was introduced in embryology to design growth of organisms, it was rapidly used in various fields, e.g. urbanism, geomorphology and even psychology. However, the use of the concept seems generally fuzzy and to have a field-specific definition for each use. We propose in this paper an epistemological study, starting with a broad interdisciplinary review and extracting essential notions linked to morphogenesis across fields. The aim is to build a consistent general meta-framework for morphogenesis, and to study the particular deviations from it. Further work may include concrete application of the framework on particular cases to operate interdisciplinary transfers of concepts, and quantitative text analysis to strengthen qualitative results.
\end{abstract}




\begin{document}


\flushbottom
\maketitle

\thispagestyle{empty}



%%%%%%%%%%%%%%%%%%%%%%%
\section*{Introduction}

At every historical period, people use the main technological advance as a metaphor to explain other phenomenon in nature. First, nature was mechanical, then electrical, and now, computational. Here we suggest that taking an alternative metaphor might allow us to better study some properties of a system, and study how the concept of morphogenesis that originated in the study of developmental biology, can be used across systems. Morphogenesis is a very powerful metaphor that is distinct from the previous three that have been very popular in history. Unlike the mechanical, electrical or computational explanations of nature, morphogenesis is not a human designed process. Morphogenesis emphasize the role of change and growth, rather than a static state. "... natural history deals with ephemeral and accidental, not eternal nor universal things ..." \cite{thompson1942growth} The goal of this paper is to study 1. the definition of the word morphogenesis in different fields. 2. fields that use similar approaches and concepts that embodies the notion of morphogenesis but does not use the word morphogenesis. 3. if approaches to study morphogenesis can be applied across different fields. A similar effort is described in~\cite{bourgine2010morphogenesis}, but it consists more on a collection of viewpoints from subjects that can be related to morphogenesis rather than an epistemological reconstruction of the notion as we propose to do. Furthermore, examples are far from exhausted and our review is thus complementary.

The rest of this paper is organized as follows : we provide first a compartmentalized review of the notion of morphogenesis across various fields, ranging from biology to social sciences, psychology and territorial sciences. A synthesis is then made and a framework as general as possible proposed. We finally discuss further developments and potential application of this epistemological analysis.


%%%%%%%%%%%%%%%%%%%%%%%
\section*{Reviews}


%%%%%%%%%%%%%%%%%%%%%%%%
%How to set up dev bio part in morpho paper?
%General aims: How is morphogenetics defined in dev bio, how does this fit the bigger picture?
%What methods are applied?
%Examples?
%Can we identify areas in which morphogenesis is not used (e.g. only chemistry is looked at, forces ignored, etc?)?
%
%Be more broad about general aspects of morphogenesis in biology, then narrow down to some specifics we know a lot about 
%
%Organoids: we can reproduce a great deal of morphogenesis, but still don't understand everything, Sato&Clevers
\subsection*{Developmental Biology}
\subsubsection*{Notion of morphogenesis in biology}
%Some founders of the field: D'Arcy Thompson (On Growth and Form, 1917) Turing (1952)
%Special issue in Science: http://www.sciencemag.org/site/special/morph/index.xhtml
% First in depth application of this theory to biology by Gierer&Meinhardt Kybernetik (1972),

\begin{itemize}
\item{Reaction-Diffusion mechanisms} 

General idea: how is information transduced, how does this information then become translated into forces in order to create shape/function changes of the developing organism?

One example of the emergence of patterns from homogeneous background is the reaction diffusion model. Using this model we can recapitulate many pattern formation mechanisms in development, such as coloration, , polarity formation and segmentation. These larger scale patterns are generated by the interaction of a few species of chemicals. Every chemical species also  undergo diffusion, production and degradation. Thus it is possible to represent this model using a system of differential equations, and certain parameters will generate stable patterns from homogeneous initial condition, where random perturbation is amplified by the system. In this model, very complex patterns can be formed with only a few species of molecules\cite{kondo2010reaction}. One of the most studied reaction diffusion model capable of producing stable patterns comprises of two types of molecules, one activator and one repressor. The difference in diffusion rate between the two molecules is what amplifies random noise in the system.\cite{gierer1972theory}\\ 

The most well studied model of reaction diffusion model explaining coloration is in zebrafish. Melanophores produce black pigment and xanthophores produce yellow pigment.\cite{nakamasu2009interactions} The interaction of melanophores and xanthophores produces the striped pattern on zebrafish. Melanophore growth is promoted by long-distance interaction with xanthophores. Short distance interactions between the two cell types inhibit each other. \\

Polarity formation in yeast division can also be explained by the reaction diffusion process of Cdc42. Cdc42 has two forms, one active membrane bound and one inactive cytoplasmic.  \cite{goryachev2008dynamics} \\

Segmentation usually involves a more complex system than the two previously discussed example, because the pattern it generates needs to be robust to ensure the correct function, and thus cannot be sensitive to variation in initial conditions. 

Evolution has also produced reaction-diffusion processes that 
\end{itemize}
% Lars : C Elegans : models of force transmission. how the form comes from the molecular perspective ; but only physics model.
% Lars: In C. elegans emergence of shape from molecular characteristics can be divided into several steps. First, in the one cell stage, there is only minimal asymmetry in the developing embryo: it is symmetric apart from the region where the sperm has entered the oocyte, which is where symmetry will broken 'properly'. This process of symmetry breaking During early development symmetry is first broken when the sperm enters the oocyte. Prior to fertilisation, any part of the oocyte could later become head or tail. After sperm entry however, the future anterior-posterior (front-rear) axis has been determined. 
%position of sperm entry is somewhat random, also not large-scale, needs robust amplification to allow reproducible development. achieved via mechano-chemical coupling (see previous section on reaction diffusion). This achieves full specification of the future head and tail by the two cell stage. breaking of left-right symmetry. D-V axis specified between 2 and 4 cell stage (EMS defines ventral), LR between four and six cell stage. This is mostly done by geometric constraints (??) and biochemical signalling, forces play a crucial role later on (and in one cell stage for symmetry breaking). A precisely regulated network gives rise to a completely determined cell lineage, reproducing animals with a near identical layout.
%  others : qualitative description. no network stuff. rq : not necessary to have networks.
%  fundamental idea. how does the shape control the function.
% Celegans totally precisely determined.



% compare the two biological approach ?
%   comparative biology ? -> many species studied.
\subsubsection*{Mechanistic modeling}
Multi-Scale Modeling in Morphogenesis: A Critical Analysis of the Cellular Potts Model (Voss-Boehme, 2012) \cite{voss2012multi}
Modeling tissue morphogenesis and cancer in 3D Yamada (2007) \cite{yamada2007modeling}
Forces in Tissue Morphogenesis and Patterning (Heisenberg, 2012) \cite{heisenberg2013forces}
A strategy for tissue self-organization that is robust to
cellular heterogeneity and plasticity (PNAS, 2015) \cite{cerchiari2015strategy} ;
Vertex Models of Epithelial Morphogenesis~\cite{fletcher2014vertex}

Force generation: microtubules, acto-myosin cortex dynamics, motors work by binding to microtubules/actin-filaments and directional 'walking'
Force transduction: adherens junctions, link neighboring cells
Mathematical frameworks: brownian ratchet model (motor proteins), active fluid theory as a relatively new theory to bridge from micro to mesoscale (Julicher/Salbreux/Bois), long range interactions/cellular potts/3D vertex models (Bielmeier 2016)
Notion of model organism, pick three: volvox, Drosophila, tissue culture/3D in vitro models 

%%%%%%%%%%%%%%%%%%%%%%%%
\paragraph{Botany}

\cite{lord1981cleistogamy} : botany

%%%%%%%%%%%%%%%%%%%%%%%%
\paragraph{Extended Phenotype}

The patterns of an ant nest are the result of ``the morphogenesis of an extended phenotype''~\cite{minter2012morphogenesis}. Here boundaries of the system are extended beyond basic organisms, as a kind of social morphogenesis. The ant nest can be seen as a larger scale organism ? self-organisation with \emph{architecture} and \emph{cognition}. Q : is the ant colony autopoietic ? Rq : the organism is the ant colony itself, see scaling lecture with body mass.


%%%%%%%%%%%%%%%%%%%%%%%
\paragraph{other}

% (Juste) I saw this paper the other day, dealing with identification of signalling processes for the morphogenesis of the heart (if I understand well, they identify proteines driving the process) -> the notion is applied to an organ only, maybe it is an interesting example to show the different scopes.

% (Lars) Cool paper! As you noted, they mainly talk about the biochemical signaling involved, not about how the actual shape of the tissue comes about. We should figure out whether this shape thing has been studied previously. This would be another nice example of how forces/geometric constraints and signaling determine size and shape and function of a body unit.
\cite{han2016coordinating}




%%%%%%%%%%%%%%%%%%%%%%%%
\subsection*{Artificial Intelligence}

% taken from Emanuele's review
% link : https://www.dropbox.com/s/agug14i106eig9f/Artificial%20Self-Assembly.pdf?dl=0

As reviewed in~\cite{crosato2014self}, the notion of \emph{Programmable Self-Assembly} seems for students of Artificial Life to be very close to the biological concept of morphogenesis : ``The greater example of Programmable Self-Assembly in nature is probably the cell organisation in multicellular organisms, which is encoded by the DNA.'' An important approach is Doursat's concept of Morphogenetic Engineering, that focuses on designing complex systems from the bottom-up. A review of the field is done in~\cite{doursat2013review} : an essential distinction between self-organization and morphogenesis that it introduces is the presence of an architecture. An example of heterogeneous swarm of particles, yielding complex architectures is described in~\cite{doursat2008programmable}. The processes of local interactions (corresponding in biology to local physical forces) and positional information through gradient propagation are both integrated in the swarm model and allow to architecture complex patterns from the bottom-up. The combination of a chemical reaction layer with an hydrodynamics layer provides also an interesting model of morphogenesis in~\cite{cussat2012synthesis}.




%%%%%%%%%%%%%%%%%%%%%%%%
\subsection*{Territorial Sciences}

The concept is used in various disciplines dealing with territories and the built environment : geography, urban planning and design, urbanism, architecture. There not seem to be a unified view nor theory among these various fields, but even in each field itself.

%%%%%%%%%%%%%%%%%%%%%%%
\paragraph{Built Environment}

Architecture and Urbanism are disciplines studying human settlements and the built environment at relatively small scales. Olsen's theory of Urban Metabolism \cite{olsen1982urban} links city morphogenesis with urban metabolism and urban ecology. The city is seen as a living organism with has different time scales of evolution (the life cycles). The study of Urban Morphology~\cite{moudon1997urban}, that focuses on morphogenetic processes, is presented as an emerging field in itself, across geography, architecture and urban planning : this view emphasis the crucial role of the form in these kind of processes.~\cite{burke1972dublin} studies the growth of a particular city on a given period of time, and attributes the evolution of urban morphology to \emph{morphogenetic agents}, i.e. people and developers. At an other scale, in architecture, a building can be seen as the results of micro-processes making sense and a particular architectural style interpreted through the use of generative shape grammars~\cite{ceccarini2001essai}. This methodology is not far from the work of C. Alexander, an illuminated architect that produced a theory of design process \cite{mehaffy2007notes}, inspired from computer science and biology and linked in some aspects to complexity. The notion of morphogenesis is in that case however quite loose, as referring to the processus of form generation in general.

\cite{dollens2014alan}

\cite{whitehand1999urban}


%%%%%%%%%%%%%%%%%%%%%%%
\paragraph{Modeling}

The Urban growth modeling literature often refer to the growth process as morphogenesis, when the scale implied allows to exhibit patterns. An example of the emergence of qualitatively different urban functions, based on Alonso-Muth model is proposed in~\cite{bonin2012modele}. \cite{makse1998modeling} studies a model of urban growth involving the local urban form. In that case the local spatial correlations induce urban structure when the cities gains new inhabitants. More heterogeneous models imply a coupling between city components and transportation networks. \cite{achibet2014model} describe a model of co-evolution between road network and urban blocks structure. At a larger scale and involving more abstract functions, \cite{raimbault2014hybrid} couples city growth with network growth, including local feedback of the form through a density constraint and global feedback of position through network centrality and accessibility to amenities. These two mechanisms are analog to the local interaction and global information diffusion flow in biology.


%%%%%%%%%%%%%%%%%%%%%%%
\paragraph{Archeology}

The morphogenesis of past human settlements viewed from Thom's Catastrophe Theory point of view, is introduced by~\cite{renfrew1978trajectory}. Sudden changes (qualitative changes, or regime shifts) have occurred at any time and can be viewed as bifurcations during the morphogenesis process. [\textit{Note : link with the notion of transition (transmondyn) that must then be more than an evolution of meta-parameters of the dynamics (non-stationarity), but where bifurcations can occur - clarify parameter evolution in Thom's theory}]




%%%%%%%%%%%%%%%%%%%%%%%
\subsection*{Social Sciences and Psychology}

Morphogenesis has been sparsely used as a useful metaphor to understand different processes in Social Sciences and various psychological fields. For example, in Developmental Psychology one can think of the relation to evolution of human cultural behavior and learning, epigenetic neural systems, and their influence on neural development and behavior throughout life , and in Clinical psychology and psychopathology we have analogies in the emergent of psychical structures and the self-organization of forms of relation with the self and the other, and of course in Neuroscience we have a plethora of morphogenetic phenomena related to  the structure of the neural nets and “hardware” of the brain.

In social psychology we have remarkable examples like the morphogenetic approach proposed by Margaret Archer applied to the problem of structure and agency, in other words, how we both shape society and are shaped by it in a dynamic interplay. Thus the morphogenetic approach offers a new understanding of social change and of the subject within it.  

Nonetheless, more than a systematic consensus in it’s usage, there is a common perception of it’s explanatory power to grasp notions that show remarkable morphogenetic concordance.

For example, it’s usage in the field of Psychoanalysis has been evoked as early as 1918 to understand the formation of psychical structures and their dynamics, the pervasive repetition of early development, and the self organization in the symptom. After an extensive review on the available bibliography contained in the database Psychoanalytic Electronic Publishing we can argue that it has been most widely used after the sixties in relation to the morphogenetic qualities of the structuration of the drive theory (for example the article Instinct Theory in the Light of Microbiology, Therese Benedek, M.D. ~\cite{benedek1973instinct}), and moreover thanks to the ideas of René Thom on Structural Stability and Morphogenesis.




% discussion with Mario : morphogenesis of ideas as qualitative emergence ?

% not sure exactly psychology, but found paper using the notion of "social morphogenesis" to explain the emergence of different cultures
\cite{straus1977societal}
% socio-psychology

% first paper : 1918

% not very used, but useful. similarity found in processes of the mind. there is morphogen in the physical biology of the brain. similarity in the psychological structure. what is the basic unit ?
% ideas, personality.
%  (45 articles)
%  <-> link literature self-org, fractals





%%%%%%%%%%%%%%%%%%%%%%%
\subsection*{History of the notion}
The study of morphogenesis started with embryology between just before 1930's. This is about the same time as Hodgkin and Lister, reported seeing red blood cells under a microscope, and less than 10 years before Dujardin's discovery of cellular movement in Amoeba. \cite{abercrombie1977concepts} Using google book, the first use of the word morphogenesis in a book is in 1871, saw a large peak in usage between 1907-1909, and continued to increase in usage until the 1990's before slowing decreasing in usage. 
The study of morphology has been tightly linked to statistical modeling. 
%Turing was a member of The Ratio Club, as was Ashby. Ashby’s Cyberneticss is consistent with Whitehead’s theory and Turing’s model, and gives an explanation as to why new equilibria so commonly emerge, rather than having chaos. (Keynes’ more jaundiced view was that we only see structure and simply ignore the chaos, treating it as ‘random’. There may be an element of truth in both views. But Ashby’s is less challenging.

% Possible correspondance with currents in history of science ?


%%%%%%%%%%%%%%%%%%%%%%%
\subsection*{Others}

\paragraph{Epistemology}

Morphogenesis is also used for the advent of new study : for example~\cite{gilbert2003morphogenesis} studies the evolution of evolutionary developmental biology through the metaphor of morphogenesis. He sees scientific ideas as interacting agents from which emerge new phenotype through differentiation processes, what is designed as the morphogenesis of the field.


\paragraph{A mathematical approach}

Ren{\'e} Thom, in \emph{Structural stability and Morphogenesis}~\cite{thom1974stabilite} has developed a theory of system dynamics, the ``catastrophe theory'', that studies in deep the impact of topological structure of phase space manifolds on a system dynamics. Let $M$ a differentiable manifold, in which system state $(m,\dot{m})$ is embedded. We assume the existence of a closed set $K$, called \emph{Catastrophe set}. The topological type of $K$ is indeed endogenously determined by system dynamics (in simple cases, it refers to the "classical" type of attractors/fixed points usually known : points, limit cycles). When $m$ encounters $K$, the system follows a \emph{qualitative} change in its form, what constitutes the basis of \emph{morphogenesis}. This abstract theory of morphogenesis is independent of the nature of the system studied, its main contribution being to classify local catastrophes that occur during morphogenesis. Differentiation and richness of patterns have thus a geometrical explanation through the topological types of catastrophes. Thom notes that at this time, the study of form has mainly be the focus of biology, but that many applications could be done in physics and geomorphology for example. He formulated the hypothesis that it is because it implies discontinuities and self-organisation, to which mathematicians were repulsive, that it was not applied easily to various fields. We can link this to the rise of complexity approaches, with complexity paradigms that slowly spreaded in various disciplines, and the study of morphogenesis seem to have followed.


\paragraph{Autopoiesis and Morphogenesis}

The notion of \emph{autopoiesis} expresses the ability for a system to reproduce itself. A basic characterization is a semi-permeable boundary produced within the system and the ability to reproduce its components. A more general definition is proposed by Bourgine and Stewart in~\cite{bourgine2004autopoiesis} : \textit{``An autopoietic system is a network of processes that produces the components that reproduce the network, and that also regulates the boundary conditions necessary for its ongoing existence as a network''}. The notion of dynamical processes is key, and could be linked to Thom's theory of morphogenesis. They furthermore introduce a definition of cognition (trigger actions as function of sensory inputs to ensure viability), and of living organism as autopoietic and cognitive, both notions being distinct~\cite{bitbol2004autopoiesis}. In that frame, for example, the arbotron~\cite{jun2005formation} is cognitive but not autopoietic. An example of link between autopoiesis and morphogenesis is shown in~\cite{niizato2010model}, where a type of Physarum organism has to play both on cell mobility and form evolution to be able to collect the food necessary for its survival. At this stage, we can postulate a strict inclusion from autopoietic systems, morphogenetic systems to self-organizing systems.


\paragraph{Co-evolution}

Since morphogenesis can be transposed to ecosystem or societies, and the components of the system are co-evolving in those cases, the existence of co-evolution may be linked with morphogenesis, as an other way of seeing the system. Symbiosis in biology can lead to very strong causalities in organism evolution (co-evolution) : this phenomenon has been designed as \emph{symbiogenesis}. The symbiosis induce an change in morphogenetic patterns of symbiotic organisms as exemplified for different species in~\cite{chapman1998morphogenesis}. Thus the strong link between morphogenesis and co-evolution (here morphogenesis designing more evolutionary paths of morphogenetic patterns, i.e. at a different time scale).






%%%%%%%%%%%%%%%%%%%%%%%
\section*{Synthesis}

%%%%%%%%%%%%%%%%%%%%%%%
\subsection*{Key notions}

\textit{Main concepts that come out from this review, and on which we think it is necessary to think. Each may be domain-dependent, and answer may vary from one field to the other ?}


\begin{itemize}
\item \textbf{Self-organisation} : morphogenesis implies self-organisation but the contrary is not necessarily true ? what aspects are specific of morphogenesis ? 
% there is not the self-replicating part (autopoiesis - life ?)
% no because more precise. bounadries ?
\item \textbf{Patterns and shape} : ``the formation of shapes'' seems to be common to all approaches to morphogenesis ?
\item \textbf{embryogenesis / tissue modeling}
\item \textbf{apoptosis}
\item \textbf{Qualitative vs Quantitative} : \textit{(Juste) I need to work more on that by I have the feeling that qualitative bifurcations are a fundamental concept in morphogenesis : e.g. differentiation of organs in biology ; emergence of differentiated urban functions}
\item \textbf{Symmetry} \textit{not sure, need to think about that} $\rightarrow$ symmetry breaking occurs at early stages (all stages ?) of morphogenesis ?
\item \textbf{Death} - in relation with life ? (read philo on that)
\item \textbf{Unit and Scale} : top-down or bottom-up, self-organized or architecture ?
% Q; do you need a fixed scale ? 
%  corals, tissue collaborating. fractal-like. idem in cities.
\item \textbf{Boundaries}
\item \textbf{Relation between Form and Function}
\end{itemize}




%%%%%%%%%%%%%%%%%%%%%%%
\subsection*{Common processes and differences}

%%%%%%%%%%%%%%%%%%%%%%%
\paragraph{From local interactions to global information flow}

% Lars : in bio, local forces/long range interactions and global info flow


%%%%%%%%%%%%%%%%%%%%%%%
\paragraph{From self-organization to morphogenesis : the notion of architecture}

Most system studied seem to have the particularity to exhibit an architecture, what would make the distinction between self-organization and morphogenesis. This idea comes from the field of morphogenetic engineering (which can be seen as a subfield of artificial intelligence). This point may be a divergence point on some fields, as for example in physical science, where the ``morphogenesis'' of terrain patterns is a self-organization in our sense. The notion of architecture may be tricky to define. A way to do it is to consider the functions of macro-levels in the system : the emergence of function at an upper level implies an architecture, which is \emph{the link between the form and the function}. Here this last concept takes all its sense and importance in regard to morphogenesis.


%%%%%%%%%%%%%%%%%%%%%%%
\subsection*{Proposition of a Meta-epistemological Framework}


\paragraph{Framework}

We propose a hierarchical organisation of concepts, that can be seen as a meta-epistemological framework, since definitions are built from synthesis of the many disciplines evoked here, and that their application in each particular discipline yields an epistemological frame. The concepts are organized the following way :

\begin{equation}
\textrm{Self-organization} \supsetneq \textrm{Morphogenesis} \supsetneq \textrm{Autopoiesis} \supsetneq \textrm{Life}
\end{equation}


each having a generic definition, elaborated from the synthesis of disciplines.

\medskip

\textbf{Definition : \textit{Self-organization}.} A system is self-organized if it exhibits weak emergence~\cite{bedau2002downward}.

\medskip

\textbf{Definition : \textit{Morphogenesis}.} A self-organized system is the result of morphogenetic processes if it exhibits an emergent architecture, in the sens of causal relations between form and function at different levels.

\medskip

\textbf{Definition : \textit{Autopoiesis and Life}.} 


\medskip

The boundary between self-organization and morphogenesis is the existence of causal links between form and function, which can be defined as \emph{architecture}~\cite{doursat2013review}, generally emergent from the bottom-up. We observe that the complexity of systems increase with notion depth, what can be loosely translated in the fact that :
\begin{itemize}
\item Emergence strength~\cite{bedau2002downward} diminishes with depth, in the sense that the number of autonomous scales increases.
\item Number of bifurcations increases~\cite{thom1974stabilite}, i.e. path-dependancy increases.
\end{itemize}


\paragraph{Application}

An ontological specification~\cite{livet2010ontology}, i.e. the definition of entities to which the notion apply, yields an application to a particular field, each one developing its own properties and level of inclusion between concepts. There is a priori no reason for a direct correspondence or equivalence of projected concepts, thus transfer of knowledge between fields may be subject to caution.


%%%%%%%%%%%%%%%%%%%%%%%
\section*{Discussion}

%%%%%%%%%%%%%%%%%%%%%%%
\subsection*{Further Developments}

\paragraph{Towards a more systematic construction}

% \item Systematize the framework : iterative construction ; systematic comparisons

Our work relies for now on a broad but not \emph{systematic} review, in the sense


%\item Algorithmic Literature Review and Text-mining : quantitative epistemology
%\end{itemize}



%%%%%%%%%%%%%%%%%%%%%%%
\subsection*{Potential Applications}


\paragraph{Transfer of Knowledge between fields}

% \item Apply to a particular subject/model



%%%%%%%%%%%%%%%%%%%%%%%
%\section{Conclusion}




%%%%%%%%%%%%
% biblio

\newpage

\bibliographystyle{apalike}
\bibliography{biblio}






%%%%%%%%%%%%%%%%%%%
%% Organisation
%%%%%%%%%%%%%%%%%%%




% Q Chenling : role of function ? Juste : necessary.

% Biological tissue : cells collaborating. over a certain level of complexity becomes more stochastic.

% Q that we will have to answer : overlaps with over concepts.

% - find synonyms ?

% - list all fields using it ?

%%%%%%%%%%%%%%
% Tasks :
%   Lars : Zebrafish
%   Mario : psycho bio [rq : mapping of words to brain areas]
%






%%%%%%%%%%%%%%%%%%%
%% Meeting 29/06
%%%%%%%%%%%%%%%%%%%


% - result : no def even in disciplines
% - review / modelography

% bio : reaction/diffusion
% does not necessarily imply life.

% MAKING MORPHO GREAT AGAIN

% Q : why sexy at a time and not anymore ? -> history of science

% bio : good review ; other fields
% Turing not cited at the beginning, until 70.
% beginning of 80s : idea : cell differentiate depending on concentration, knows what to do when knows where it is. Both are true with Turing : two combined gradients.

% Q Marius : self-orga impossible with top-down control
%-> actually seems to depend on the application. The imbrication of concepts depends on the fields/subjects etc. precise that : epistemological result ?
%(ex boundary / autopoiesis)



%%%%%%%%%%%%%%%%%%%%




\end{document}










%%%%%%%%%%%%%%%%%%%%%%%%%%%%%%%
%% TEMPLATES



%\section*{Introduction}

%The Introduction section, of referenced text\cite{Figueredo:2009dg} expands on the background of the work (some overlap with the Abstract is acceptable). The introduction should not include subheadings.

%\section*{Results}

%Up to three levels of \textbf{subheading} are permitted. Subheadings should not be numbered.

%\subsection*{Subsection}

%Example text under a subsection. Bulleted lists may be used where appropriate, e.g.

%\begin{itemize}
%\item First item
%\item Second item
%\end{itemize}

%\subsubsection*{Third-level section}
 
%Topical subheadings are allowed.

%\section*{Discussion}

%The Discussion should be succinct and must not contain subheadings.

%\section*{Methods}

%Topical subheadings are allowed. Authors must ensure that their Methods section includes adequate experimental and characterization data necessary for others in the field to reproduce their work.

%\bibliography{sample}

%\noindent LaTeX formats citations and references automatically using the bibliography records in your .bib file, which you can edit via the project menu. Use the cite command for an inline citation, e.g.  \cite{Figueredo:2009dg}.

%\section*{Acknowledgements (not compulsory)}

%Acknowledgements should be brief, and should not include thanks to anonymous referees and editors, or effusive comments. Grant or contribution numbers may be acknowledged.

%\section*{Author contributions statement}

%Must include all authors, identified by initials, for example:
%A.A. conceived the experiment(s),  A.A. and B.A. conducted the experiment(s), C.A. and D.A. analysed the results.  All authors reviewed the manuscript. 

%\section*{Additional information}

%To include, in this order: \textbf{Accession codes} (where applicable); \textbf{Competing financial interests} (mandatory statement). 

%The corresponding author is responsible for submitting a \href{http://www.nature.com/srep/policies/index.html#competing}{competing financial interests statement} on behalf of all authors of the paper. This statement must be included in the submitted article file.

%\begin{figure}[ht]
%\centering
%\includegraphics[width=\linewidth]{stream}
%\caption{Legend (350 words max). Example legend text.}
%\label{fig:stream}
%\end{figure}

%\begin{table}[ht]
%\centering
%\begin{tabular}{|l|l|l|}
%\hline
%Condition & n & p \\
%\hline
%A & 5 & 0.1 \\
%\hline
%B & 10 & 0.01 \\
%\hline
%\end{tabular}
%\caption{\label{tab:example}Legend (350 words max). Example legend text.}
%\end{table}

%Figures and tables can be referenced in LaTeX using the ref command, e.g. Figure \ref{fig:stream} and Table \ref{tab:example}.






